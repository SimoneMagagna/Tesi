\appendix
\newpage
%**************************************************************
% GLOSSARIO
%**************************************************************
\chapter{Glossario}

\section*{A}

\begin{itemize}
	\item \hypertarget{auth}{\textbf{Authoring:}} gli applicativi d'autore sono quei software verticali che consentono la realizzazione di una comunicazione multimediale, articolata e riproducibile su personal computer. L'intento è quello di poter produrre e veicolare contenuti (immagini statiche, animazioni grafiche, filmati video, commenti sonori, effetti audio e altro) su supporti come CD-ROM, DVD, via web, ma anche attraverso un circuito, chiuso o aperto, di display distribuiti e connessi tra loro in rete.
\end{itemize}

\begin{itemize}
	\item \hypertarget{ar}{\textbf{Augment reality:}} arricchimento della percezione sensoriale umana mediante informazioni, in genere manipolate e convogliate elettronicamente, che non sarebbero percepibili con i cinque sensi.
\end{itemize}

\section*{B}

\begin{itemize}
	\item \hypertarget{bg}{\textbf{Bug tracking:}}  applicativo software usato generalmente dai programmatori per tenere traccia delle segnalazioni di bug all'interno dei software, in modo che tali errori siano mantenuti sotto controllo, con una descrizione della riproducibilità e dei dettagli ad essi correlati.
\end{itemize}

\section*{C}

\begin{itemize}
	\item \hypertarget{cac}{\textbf{Command-and-control:}} principio di \textit{management} dove si afferma il mantenimento dell'autorità in un processo decisionale distribuito.
	
	\item \hypertarget{cc}{\textbf{Cloud computing:}} paradigma di erogazione di risorse informatiche, come l'archiviazione, l'elaborazione o la trasmissione di dati, caratterizzato dalla disponibilità on demand attraverso Internet a partire da un insieme di risorse preesistenti e configurabili.
	
	\item \hypertarget{cms}{\textbf{Content management system:}}  è uno strumento software, installato su un server web, il cui compito è facilitare la gestione dei contenuti di siti web, svincolando il webmaster da conoscenze tecniche specifiche di programmazione web.
	
	\item \hypertarget{cs}{\textbf{Container software:}}  soluzione al problema di come ottenere il software eseguibile in modo affidabile quando viene spostato da un ambiente informatico all'altro. Sono costituiti da un intero ambiente di runtime: un'applicazione, oltre a tutte le sue dipendenze, librerie e altri file binari e file di configurazione necessari per eseguirlo, impacchettati in un unico pacchetto.
	 
\end{itemize}

\section*{D}

\begin{itemize}
	\item \hypertarget{dep}{\textbf{Deployment:}} consegna o rilascio al cliente, con relativa installazione e messa in funzione o esercizio, di una applicazione o di un sistema software tipicamente all'interno di un sistema informatico aziendale.
\end{itemize}

\section*{E}

\section*{F}

\begin{itemize}
	\item \hypertarget{fw}{\textbf{Framework:}} architettura logica di supporto (spesso un'implementazione logica di un \textit{design pattern}) su cui un software può essere progettato e realizzato.
\end{itemize}

\section*{G}

\section*{H}

\section*{I}

\begin{itemize}
	\item \hypertarget{it}{\textbf{Issue tracking:}} pacchetto software che gestisce e mantiene liste di problemi im maniera organizzata.
\end{itemize}

\section*{J}

\section*{K}

\section*{L}

\section*{M}

\section*{N}

\section*{O}

\begin{itemize}
	\item \hypertarget{ops}{\textbf{Operations:}} funzioni di un'impresa coinvolte nella messa a disposizione per il cliente di un determinato prodotto o servizio.  
\end{itemize}

\section*{P}

\section*{Q}

\section*{R}

\begin{itemize}
	\item \hypertarget{ret}{\textbf{Retailing:}} o vendita al dettaglio, costituisce l'ultimo anello della catena di distribuzione. Il venditore al dettaglio (negozio, supermercato, eccetera) acquista quantità, relativamente elevate, di merce dal produttore o da un grossista e rivende quantità più contenute ai consumatori per ottenere un profitto.
\end{itemize}

\begin{itemize}
	\item \hypertarget{ref}{\textbf{Refactoring:}} tecnica strutturata per modificare la struttura interna di porzioni di codice senza modificarne il comportamento esterno, applicata per migliorare alcune caratteristiche non funzionati del software.
	
	\item \hypertarget{rep}{\textbf{Repository:}} è un ambiente di un sistema informativo, in cui vengono gestiti i metadati, attraverso tabelle relazionali; l'insieme di tabelle, regole e motori di calcolo tramite cui si gestiscono i metadati prende il nome di metabase.
	
\end{itemize}

\section*{S}

\begin{itemize}
	\item \hypertarget{sdk}{\textbf{Software development kit:}}  insieme di strumenti per lo sviluppo e la documentazione di software.
	
	\item \hypertarget{sms}{\textbf{Social management system:}} software che permette la gestione dei propri social network, collezionando contenuti presenti in essi o interagendovi automaticamente effettuando operazioni desiderate.
	
	\item \hypertarget{sh}{\textbf{Stakeholder:}} soggetto direttamente o indirettamente coinvolto in un progetto o in un'attività di un'azienda.
\end{itemize}

\section*{T}

\begin{itemize}
	\item \hypertarget{test}{\textbf{Testing:}} il \textit{software testing} è un'attività di investigazione condotta per fornire alle parti interessate informazioni sulla qualità dal prodotto o del servizio in prova.
\end{itemize}

\section*{U}

\begin{itemize}
	\item \hypertarget{uml}{\textbf{UML:}} è un linguaggio di modellazione e specifica basato sul paradigma orientato agli oggetti. UML consente di costruire modelli object-oriented per rappresentare domini di diverso genere. Nel contesto dell'ingegneria del software, viene usato soprattutto per descrivere il dominio applicativo di un sistema software e/o il comportamento e la struttura del sistema stesso. Il modello è strutturato secondo un insieme di viste che rappresentano diversi aspetti della cosa modellata (funzionamento, struttura, comportamento, e così via), sia a scopo di analisi che di progetto, mantenendo la tracciabilità dei concetti impiegati nelle diverse viste. Oltre che per la modellazione di sistemi software, UML viene non di rado impiegato per descrivere domini di altri tipi, come sistemi hardware, strutture organizzative aziendali, processi di business.
	\item \hypertarget{ut}{\textbf{Unit testing:}} per test di unità si intende l'attività di \textit{testing} di singole unità software. Per unità si intende il minimo componente di un programma dotato di funzionamento autonomo. 
\end{itemize}

\section*{V}

\begin{itemize}
	\item \hypertarget{vr}{\textbf{Virtual reality:}} realtà simulata attraverso dispositivi elettronici, come visori, cuffie e sensori di movimento.
\end{itemize}

\section*{W}

\section*{X}

\section*{Y}

\section*{Z}