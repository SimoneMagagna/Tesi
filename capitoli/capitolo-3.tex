\newpage
%**************************************************************
\chapter{Il progetto di e-commerce VR}
\label{cap:ilprogettoe-commercevr}

In questo capitolo andrò a trattare nel dettaglio tutte le fasi dello stage che hanno portato allo sviluppo del progetto.

\section{Pianificazione del lavoro}

In questa sezione tratterò della pianificazione del lavoro effettuata assieme al mio tutor, delle fasi che l'hanno caratterizzata e del ciclo di vita adottato. 

\section{Ricerca e sperimentazione}

In questa sezione descriverò la fase di ricerca e sperimentazione delle tecnologie utilizzate, inizialmente a me sconosciute. Ho deciso di dedicare una sezione a questa fase perché ha avuto una rilevante importanza all'interno del mio stage e rappresenta uno dei principali obbiettivi aziendali.

\section{Tecnologie adottate}

In questa sezione descriverò come le ricerche e le sperimentazioni effettuate mi hanno portato a scegliere un particolare stack tecnologico.

\section{Analisi dei requisiti}

All'interno di questa sezione tratterò dell'attività di analisi dei requisiti che il team ha effettuato prima della progettazione e dello sviluppo del progetto.

\subsection{Caratteristiche degli utenti}

In questa sottosezione descriverò le tipologie di utenti previsti per l'applicazione.

\subsection{Casi d'uso}

In questa sottosezione elencherò tutti i casi d'uso previsti per l'applicazione.

\subsection{Requisiti}

In questa sottosezione elencherò tutti i requisiti estratti che l'applicazione soddisfa.

\section{Progettazione}

In questa sezione andrò a descrivere le più importanti fasi di progettazione.

\subsection{Portabilità dell'applicazione}

In questa sottosezione tratterò di come la progettazione del software sia stata ampiamente influenzata dalla volontà di portabilità dell'applicazione su tutti i dispositivi VR Android.

\subsection{Usabilità dell'applicazione}

In questa sottosezione descriverò gli studi effettuati riguardo l'usabilità dell'applicazione VR e delle scelte di progettazione che hanno portato tali studi.

\subsection{Costruzione della scena 3D}

In questa sezione andrò a descrivere le fasi di progettazione della scena 3D presente nell'applicazione e visibile tramite dispositivo VR.

\subsection{Interazione con gli oggetti all'interno della scena} 

All'interno di questa sottosezione parlerò della progettazione riguardante le modalità di interazione tra il visore VR e gli oggetti presenti all'interno della scena.

\subsection{Progettazione e integrazione con AWS API Gateway}

All'interno di questa sezione tratterò della progettazione riguardante l'API Mock creata tramite AWS API Gateway e della sua integrazione con l'applicazione.

\section{Sviluppo}

In questa sezione andrò a descrivere in dettaglio lo sviluppo delle più significative e peculiari funzionalità dell'applicazione.

\subsection{Sviluppo degli oggetti interattivi}

In questa sottosezione descriverò come si costruiscono degli oggetti interattivi in Unity per i dispositivi VR.

\subsection{Creazione a runtime di oggetti interattivi}

In questa sottosezione tratterò della creazione a runtime di oggetti interattivi in Unity.

\subsection{Dati persistenti attraverso le scene}

In questa sezione spiegherò come si costruiscono oggetti persistenti che vivono attraverso le scene.

\subsection{Unity e il protocollo HTTP}

In questa sottosezione parlerò di come Unity si integri con il protocollo HTTP.

\subsection{Creazione e parsing di oggetti JSON in Unity}

In questa sottosezione parlerò di come si creino e si manipolino oggetti JSON in Unity.

\section{Verifica e validazione}

All'interno di questa sezione parlerò della fase di verifica e validazione effettuata per questo progetto.
