\newpage
%**************************************************************
% CAPITOLO 4
%**************************************************************
\chapter{Analisi retrospettiva}
\label{cap:analisiretrospettiva}

\section{Bilancio dei risultati rispetto agli obiettivi prefissati}

Come affermato dal tutor aziendale, gli obiettivi che l'azienda aveva fissato nel piano di lavoro rappresentavano le aspettative che gli \textit{stakeholder}\hyperlink{sh}{\ped{G}} avevano sulla tecnologia \textit{VR}\ped{\hyperlink{vr}{G}}, senza però essere avvalorate da alcuna conoscenza specifica. Nemmeno il \textit{team} R\&D, reparto di ricerca e sviluppo, era riuscito ad approfondirne lo studio, a causa di impegni aziendali più prioritari. Dunque, non erano certi che tali obiettivi potessero essere soddisfatti, lasciandomi così completa libertà sia per quanto riguarda le tecnologie da adottare sia per quanto riguarda le tempistiche. Questa libertà mi ha permesso di raggiungere molti degli obiettivi primari, che sono andati definendosi maggiormente con l'aumento della mia personale conoscenza sulle tecnologie:
\begin{itemize}
	\item \textbf{Prototipo scena 3D:} come discusso nel paragrafo 3.2.3 abbiamo deciso di costruire la scena attraverso una foto a 360 gradi di una stanza applicata ad una sfera inversa invece che modellare l'intera scena con oggetti 3D;
	\item \textbf{Progettazione e sviluppo di oggetti e comportamento di essi nello spazio 3D:} come discusso nei paragrafi 3.2.3 e 3.2.4 non ho modellato gli oggetti attraverso l'editor grafico di \textit{Unity}, ma ho posto dei pannelli interattivi invisibili davanti agli oggetti che compongono la foto a 360 gradi;
	\item \textbf{Progettazione e sviluppo integrazione tra sistema di VE e e-commerce:} come discusso nel paragrafo 3.2.5 non ho integrato l'applicazione con un vero \textit{e-commerce} ma con un API creata tramite \textit{Amazon Web Service};
	\item \textbf{Approfondimento di user interactions (Leap Motion, eccetera):} presa la decisione della strada \textit{mobile}, le uniche \textit{user interaction} possibili erano tramite i due dispositivi \textit{VR}\ped{\hyperlink{vr}{G}} presenti in azienda: \textit{Samsung Gear VR} e \textit{Google Cardboard};
	\item \textbf{Studio e prototipazione del possibile processo di acquisto all’interno di un VE (registrazione, checkout, pagamenti eccetera):} come discusso nel paragrafo 3.2.2 il \textit{team} ha scelto di permettere all'utente una registrazione ad un normale \textit{e-commerce}, con la possibilità di immettere i dati di pagamento, prima dell'utilizzo dell'applicazione.
\end{itemize} 

\begin{table}
	\centering
	\label{tabella-obiettivi}
	\begin{tabular}{| p{6cm} | p{6cm} |}
		\hline
		\textbf{Obiettivo} & \textbf{Realizzazione} \\ \hline
		 \textbf{Minimo:} Studio delle tecnologie disponibili in ambito \textit{VR}\ped{\hyperlink{vr}{G}} e stesura di un documento riassuntivo che offra un \textit{overview} dello stato attuale della realtà aumentata. &  \textbf{Obbiettivo realizzato:} ho redatto un documento che riporta tutte le tecnologie VR presenti nel mercato e i rispettivi stack tecnologici.\\ \hline
		 \textbf{Minimo:} Progettazione e sviluppo di un ambiente virtuale con: una scena e oggetti definiti, un comportamento associato agli oggetti e un prototipo di \textit{user interaction}. & \textbf{Obbiettivo realizzato:} come esposto nei paragrafi 3.2.3, 3.2.4, 3.3.2 e 3.3.4, ho realizzato la scena 3D, gli oggetti e il loro comportamento e implementato una \textit{user interaction}. \\ \hline
		 \textbf{Minimo:} Scambio di informazioni di base tra l'applicazione e un sistema di \textit{e-commerce}. & \textbf{Obiettivo parzialmente realizzato:} ho in parte realizzato l'integrazione tra l'applicazione e un \textit{e-commerce} come descritto nella sezione 3.2.5.\\ \hline
		 \textbf{Massimo:} Studio e prototipazione di diversi modelli di \textit{user interaction} con l'ambiente e con gli oggetti finalizzati alla presentazione di un bene vendibile. & \textbf{Obbiettivo parzialmente realizzato:} presa la decisione di intraprendere la strada \textit{mobile}, i possibili modelli di \textit{user interaction} erano due: \textit{Samsung Gear VR} e \textit{Google Cardboard}. Come esposto nel paragrafo 3.2.1, non ho potuto implementare completamente la portabilità dell'applicazione nei due ambienti di studio.\\ \hline
		 \textbf{Massimo:} Studio e implementazione di possibili nuovi processi di acquisto in ambito \textit{VR}\ped{\hyperlink{vr}{G}}. & \textbf{Obbiettivo parzialmente realizzato:} come esposto nella sezione 3.2.2, il \textit{team} ha studiato studiati diversi processi d'acquisto in ambito \textit{VR}\ped{\hyperlink{vr}{G}} ed uno di questi è stato in parte realizzato.\\ \hline
	\end{tabular}
	\caption{Tabella degli obiettivi minimi e massimi aziendali discussi nella sezione 2.2.2, con affianco il loro grado di realizzazione}
\end{table}
\FloatBarrier

Nonostante non sia riuscito a portare a termine alcuni obiettivi, mi ritengo pienamente soddisfatto per i molti che invece sono riuscito realizzare. La tecnologia che ho dovuto utilizzare per la realizzazione di questo progetto, era a me completamente sconosciuta e molti degli obiettivi non realizzati, come discusso nel capitolo 3, sono risultati essere impossibili da portare a termine a causa dell'immaturità della tecnologia \textit{VR}\ped{\hyperlink{vr}{G}}. \\ 
Il più importante obiettivo di questo stage era sperimentare le potenzialità e i limiti della realtà virtuale applicata all'ambito \textit{e-commerce} e tale obiettivo è stato completamente portato a termine, come confermatomi dal tutor aziendale.

\section{Bilancio formativo}

Il bagaglio culturale che questa attività di stage mi ha permesso di creare è formato sostanzialmente da due parti. \\
La prima è costituita dalle \textbf{conoscenze tecnologiche} apprese, le quali non fanno parte del normale percorso universitario. La costruzione di ambienti e oggetti tridimensionali e lo sviluppo del loro comportamento, sono campi dell'informatica davvero interessanti da studiare e sperimentare, tanto che vi sono scuole che offrono interi master sull'argomento. Ad oggi, la computer grafica non è più circoscritta solo all'ambito videoludico, ma viene utilizzata in maniera massiccia in molti altri campi come quello cinematografico e pubblicitario. Essa, dunque, rappresenta un notevole valore aggiunto per quanto riguarda il curriculum personale. La seconda tecnologia che lo stage mi ha permesso di studiare, e che non rientra nei corsi universitari, è \textit{API Gateway} di \textit{Amazon Web Service}. Amazon si sta affermando come una delle più importanti piattaforme di servizi di \textit{cloud computing}\hyperlink{cc}{\ped{G}} e lo sviluppo di un API tramite questa tecnologia mi ha permesso di comprendere meglio la natura della comunicazione tra i servizi di front-end e back-end. Infine, anche se gli strumenti per lo sviluppo \textit{VR}\ped{\hyperlink{vr}{G}} sono risultati essere in parte incompleti, la realtà virtuale è sicuramente una delle tecnologie emergenti e più influenti dell'ultimo anno. Molte gradi aziende come Sony, Microsoft, Samsung e Google stanno investendo molto in questo settore, destinato a cambiare radicalmente la nostra quotidiana esperienza multimediale e videoludica. Poterne studiare il processo evolutivo e le sue attuali potenzialità è stato per me davvero affascinate e coinvolgente. \\
La seconda parte che va a formare il bagaglio culturale creato con l'attività di stage è sicuramente \textbf{l'esperienza aziendale}, parte che tra le due ritengo la più importante. La collaborazione con i colleghi, il rapporto con il datore di lavoro e gli \textit{stakeholder}\hyperlink{sh}{\ped{G}}, il rispetto degli orari, della sicurezza e della regolamentazione costituiscono un'esperienza fondamentale per uno studente di un indirizzo orientato al mondo del lavoro. Poter usufruire di un servizio universitario, che garantisce un'esperienza lavorativa all'interno del percorso di studi, permette a tutti gli studenti che entrano per la prima volta nel mondo del lavoro di conoscere già le dinamiche generali aziendali così da poter fin da subito adattarvisi. \\
Ritengo dunque il bilancio formativo davvero positivo e posso affermare che lo stage rappresenta una delle più importanti attività svolte nel mio personale percorso universitario.

\section{Analisi critica del rapporto formativo tra stage e corso di laurea}

Sono molte le conoscenze apprese durante il percorso universitario che mi hanno aiutato nello sviluppo di questo progetto. \\
Vi è però un'ambito dell'informatica che purtroppo non viene trattato e la sua mancanza ha influito negativamente nel mio personale progetto di stage, ovvero la \textbf{computer grafica}. Comprendo la forte difficoltà nel creare un corso di pochi mesi su un argomento così vasto e mutevole. Essa, oltretutto, è una materia davvero settoriale e pochi sono i punti di contatto con le altre materie informatiche. Ma trattarne anche solo i principi di base, costituirebbe un notevole valore aggiunto al corso di studi, se non addirittura all'intera Università. Molti studenti oggi vengono attratti dall'informatica perché appassionati di hardware e del mondo videoludico. Un corso, anche opzionale, che tocchi tali temi permetterebbe allo studente appassionato di proseguire con molto più impegno e dedizione il percorso di studi scelto. \\
Un'altro argomento fondamentale che viene solamente introdotto dai corsi universitari e che invece ho imparato a padroneggiare grazie allo stage è la \textbf{comunicazione tra i servizi di front-end e back-end}. Tramite lo sviluppo di un API e della comunicazione tra essa e l'applicazione in \textit{Unity}, ho potuto comprendere pienamente i concetti di back-end e front-end e come avviene lo scambio di informazioni tra essi tramite il protocollo HTTP. Questi due concetti rappresentano il punto cardine di tutte le attuali applicazioni e servizi informatici ed essere esplicitamente illustrati, durante i corsi di sviluppo web, aiuterebbe non poco lo studente nella realizzazione dei molti progetti richiesti. 

\section{Valutazioni personali}

Fin dagli albori della contemporanea realtà virtuale, ho sempre pensato che l'utenza non fosse ancora pronta per una tale rivoluzione multimediale. Da subito essa ha ricevuto numerose critiche da chi sostiene sia solo un mezzo per il totale distacco dalla realtà. Oltretutto, è lungi da offrire una qualità grafica al pari degli attuali monitor e televisori e, a causa della vertiginosa crescita tecnologica degli ultimi anni, è ormai difficile accogliere una nuova tecnologia che non sia in tutto migliore di quella precedente. Per poter sperimentare la massima esperienza \textit{VR}\ped{\hyperlink{vr}{G}} ad oggi possibile, è necessario un hardware di fascia alta, hardware non accessibile all'utenza media, l'unica in grado di trasformare una nuova tecnologia in uno standard. Inoltre i dispositivi meno costosi sono troppo lontani dall'attuale qualità multimediale e investire in loro con nessuna certezza di miglioramento è davvero pionieristico. Infine, il lavoro degli sviluppatori, che si affacciano per la prima volta alla realtà virtuale, non è decisamente agevolato data l'immaturità degli \textit{SDK}\ped{\hyperlink{sdk}{G}} disponibili e la totale mancanza di uno standard. \\ 
Tutto ciò porterebbe ad una visione negativa della realtà virtuale, visione, a mio parere, davvero miope. Poter vivere l'esperienza \textit{VR}\ped{\hyperlink{vr}{G}} mi ha reso consapevole del fascino che essa può offrire e delle emozioni che è in grado di regalare. Cambiare stanza, regione o Paese, immergersi completamente all'interno dell'azione di un film e interagire con l'ambiente circostante per conoscerne in tempo reale le informazioni, da oggi è possibile solamente indossando un visore. \\
La realtà virtuale è sicuramente ancora in fase embrionale, fase che durerà a lungo. Ma sono convinto che essa inesorabilmente entrerà nella nostra quotidianità, cambiando radicalmente il nostro modo di vivere la multimedialità è il nostro rapporto con la realtà. Sta ad ognuno di noi trovarne il giusto utilizzo.  