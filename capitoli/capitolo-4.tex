\newpage
%**************************************************************
% CAPITOLO 4
%**************************************************************
\chapter{Analisi retrospettiva}
\label{cap:analisiretrospettiva}

\section{Bilancio dei risultati rispetto agli obiettivi prefissati}

Come affermato dal tutor aziendale, gli obbiettivi che l'azienda aveva fissato nel piano di lavoro rappresentavano le aspettative che gli \textit{stakeholder}\hyperlink{sh}{\ped{G}} avevano sulla tecnologia \textit{VR}\ped{\hyperlink{vr}{G}} senza essere avvalorate da alcuna conoscenza specifica. Nemmeno il \textit{team} R\&D, reparto di ricerca e sviluppo, era riuscito ad approfondirne lo studio, a causa di impegni aziendali più prioritari. Dunque, non erano certi che tali obbiettivi potessero essere soddisfatti, lasciandomi così completa libertà sia per quanto riguarda le tecnologie da adottare sia per quanto riguarda le tempistiche. Questa libertà mi ha permesso di raggiungere molti degli obiettivi primari, che sono andati definendosi maggiormente con l'aumento della mia personale conoscenza sulle tecnologie e in base alle decisioni prese nelle attività precedenti. \\
Gli obbiettivi settimanali, presenti nel piano di lavoro, che hanno trovato una maggiore definizione durante il percorso del progetto sono:

\begin{itemize}
	\item \textbf{Prototipo scena 3D:};
	\item \textbf{Progettazione e sviluppo di oggetti e comportamento di essi nello spazio 3D:};
	\item \textbf{Progettazione e sviluppo integrazione di prodotto tra sistema di VE e e-commerce:};
	\item \textbf{Approfondimento di user interactions (Leap Motion, eccetera):};
	\item \textbf{Studio e prototipazione del possibile processo di acquisto all’interno di un VE (registrazione, checkout, pagamenti eccetera):}.
\end{itemize} 

La seguente tabella riporta il riepilogo degli obbiettivi minimi e massimi aziendali, discussi nella sezione 2.2.2, riportando per ognuno se è stato soddisfatto o meno:

\begin{table}
	\centering
	\label{tabella-obbiettivi}
	\begin{tabular}{| p{6cm} | p{6cm} |}
		\hline
		\textbf{Obiettivo} & \textbf{Realizzazione} \\ \hline
		 \textbf{Minimo:} Studio delle tecnologie disponibili in ambito \textit{VR}\ped{\hyperlink{vr}{G}} e stesura di un documento riassuntivo che offra un \textit{overview} dello stato attuale della realtà aumentata. &  \textbf{Obbiettivo realizzato:} è stato redatto un documento che riporta tutte le tecnologie VR presenti nel mercato e i rispettivi stack tecnologici.\\ \hline
		 \textbf{Minimo:} Progettazione e sviluppo di un ambiente virtuale con: una scena e oggetti definiti, un comportamento associato agli oggetti, un prototipo di \textit{user interaction}. & \textbf{Obbiettivo realizzato:} come esposto nei paragrafi 3.2.3, 3.2.4, 3.3.2 e 3.3.4 gli obbiettivi di realizzazione della scena, degli oggetti e del loro comportamento e della \textit{user interaction} sono stati portati a termine. \\ \hline
		 \textbf{Minimo:} Scambio di informazioni di base tra l'applicazione e un sistema di \textit{e-commerce}. & \textbf{Obiettivo parzialmente realizzato:} l'integrazione con un \textit{e-commerce} è stata in parte realizzata come descritto nella sezione 3.2.5.\\ \hline
		 \textbf{Massimo:} Studio e prototipazione di diversi modelli di \textit{user interaction} con l'ambiente e con gli oggetti finalizzati alla presentazione di un bene vendibile. & \textbf{Obbiettivo parzialmente realizzato:} presa la decisione di intraprendere la strada \textit{mobile}, i possibili modelli di \textit{user interaction} erano due: \textit{Samsung Gear VR} e \textit{Google Cardboard}. Come esposto nel paragrafo 3.2.1, la portabilità dell'applicazione nei due ambienti di studio non è stata completamente implementabile.\\ \hline
		 \textbf{Massimo:} Studio e implementazione di possibili nuovi processi di acquisto in ambito \textit{VR}\ped{\hyperlink{vr}{G}}. & \textbf{Obbiettivo parzialmente realizzato:} come esposto nella sezione 3.2.2 sono stati studiati diversi processi d'acquisto in ambito \textit{VR}\ped{\hyperlink{vr}{G}} ed uno di questi è stato in parte realizzato.\\ \hline
	\end{tabular}
	\caption{Tabella degli obbiettivi minimi e massimi aziendali}
\end{table}
\FloatBarrier


\section{Bilancio formativo}

In questa sezione analizzerò le conoscenze, le abilità e le competenze apprese durante l'attività di stage.

\section{Analisi critica del rapporto formativo tra stage e corso di laurea}

In questa sezione discuterò quali conoscenze, apprese durante lo stage, ritengo debbano essere integrate nel corso di laurea.

\section{Valutazioni personali}

In questa sezione effettuerò delle valutazioni personali riguardo al progetto e allo stage.