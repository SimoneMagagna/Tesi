\newpage
%**************************************************************
% CAPITOLO 2
%**************************************************************
\chapter{Il quadro strategico}
\label{cap:ilquadrostrategico}

\section{Strategie aziendali di stage}

L'azienda The White Dog s.r.l. accoglie e offre l'attività di stage per due principali motivi:

\begin{itemize}
	\item \textbf{Sperimentazione su progetti innovativi:} data la forte propensione alla ricerca dell'azienda, esistono numerosi campi che essa vorrebbe esplorare ma che a causa di altri progetti più prioritari e scarsità di tempo non può studiare. Offre quindi allo studente universitario un progetto di ricerca e sviluppo su tecnologie innovative ed interessanti, non pretendendo alcun risultato da subito inseribile nel mercato. Questo permette allo studente di vivere l'esperienza dello stage in piena libertà e serenità, riuscendo così a portare un notevole valore aggiunto personale che l'azienda è ben felice di accogliere;
	\item \textbf{Valutazione dello stagista:} l'azienda è in continua crescita e necessita di nuovo personale preparato e soprattutto capace di lavorare in costante sintonia col gruppo. Lo stage universitario permette all'azienda di scoprire persone che soddisfano questi due importanti requisiti per una futura assunzione.
\end{itemize}

Da parte sua The White Dog s.r.l. offre molto agli stagisti. I tutor aziendali supportano lo studente per tutto il periodo lavorativo, consigliandolo sia per quanto riguarda il piano di lavoro, sia sulle tecnologie da utilizzare sia effettuando proficue discussioni in vista della relazione finale. Allo studente viene offerto un ambiente di lavoro accogliente e strumenti aggiornati e all'avanguardia, supportandolo anche economicamente prevedendo un rimborso spese.

\section{Il progetto di stage proposto}

Il progetto propostomi nasce dalla costante volontà aziendale di ricercare nuove metodologie di interazione da proporre agli utenti dei suoi \textit{e-commerce} in ottica \textit{omni-channel}. \\ 
Il modello \textit{omni-channel} si sta lentamente ma inesorabilmente affermando come principale modello di \textit{retailing}\ped{\hyperlink{ret}{G}} a livello globale e si basa sulle seguenti caratteristiche:

\begin{itemize}
	\item Concezione e management unitario della distribuzione;
	\item Processi basati sull'interazione, comunicazione e interdipendenza tra i \textit{team} dedicati ai singoli canali;
	\item Approccio dinamico al consumatore, che richiede un monitoraggio in tempo reale delle evoluzioni dei comportamenti di acquisto e delle risposte alle iniziative promosse;
	\item Predisposizione di adeguati strumenti IT e marketing in grado di sfruttare ed assecondare il fenomeno della cross-canalità dei processi di acquisto;
	\item Impatto competitivo decisivo delle scelte organizzative e di investimento nell'IT e nel marketing digitale;
	\item Impiego di indicatori di prestazioni e di sistemi di \textit{monitoring} adeguati al nuovo contesto.
\end{itemize}

Lo studio sulle nuove tecnologie e strumentazioni presenti sul mercato, ha portato l'azienda a considerare la realtà virtuale una tecnologia adatta all'obiettivo della omni-canalità. Nasce così l'idea di un \textit{e-commerce VR}, progetto in grado di colmare, in parte, quel divario che da sempre ha distanziato \textit{store} virtuale e negozio fisico.

\label{Omni-channel}
\begin{figure}[ht]
	\begin{center}
		\includegraphics[scale=0.47]{omni-channel}
		\caption{Schema rappresentativo della differenze tra \textit{single-channel}, \textit{multi-channel}, \textit{cross-channel} e \textit{omni-channel}}
	\end{center}
\end{figure}
\FloatBarrier

L'obbiettivo di stage è, dunque, un'esplorazione tecnologica nel campo della \textit{vitual reality}\ped{\hyperlink{vr}{G}}. Il progetto mira ad arrivare ad un prototipo di \textit{vitual showroom} dove poter esplorare ed interagire con i prodotti e permetterne l'acquisto. \\
Il progetto era stato inizialmente diviso in due parti, per due differenti studenti:
\begin{itemize}
	\item La prima parte riguardava la progettazione e realizzazione del movimento in uno spazio 3D ed interazione con gli oggetti;
	\item La seconda parte trattava invece la progettazione e realizzazione in un'interfaccia di presentazione del prodotto, con integrazione al processo di acquisto mediante l'uso di sistemi \textit{cloud} esterni.
\end{itemize}
Purtroppo, nessun altro studente oltre a me ha aderito al progetto, dunque si è dovuto rivisitare l'obbiettivo di stage. Dopo una riunione effettuata prima dell'inizio dello stage con il mio tutor aziendale, abbiamo deciso di mantenere intatti tutti gli obiettivi di ricerca, abbassando il livello qualitativo richiesto. Questo perché lo scopo ultimo di questo stage non era sviluppare un'applicazione o un servizio immediatamente vendibile, ma di studiare le potenzialità e i limiti di questa nuova tecnologia. Questa volontà è stata dettata anche dal fatto che il \textit{team} aziendale inizialmente non aveva alcuna certezza che la tecnologia \textit{VR}\ped{\hyperlink{vr}{G}} fosse applicabile al mondo \textit{e-commerce}.

\subsection{Piano di lavoro proposto}

\subsubsection{Piano temporale}

In accordo col tutor aziendale, la durata massima dello stage è stata fissata a 320 ore, divise in 8 settimane lavorative di 5 giorni, 8 ore al giorno. \\
Il piano lavorativo è stato dunque pianificato per settimana nel seguente modo:

\begin{itemize}
	\item \textbf{Settimana 1:} settimana dedicata completamente alla ricerca, per colmare il \textit{deficit} culturale personale e aziendale sulle tecnologie \textit{VR}\ped{\hyperlink{vr}{G}}. Le attività principali previste sono: analisi dei requisiti funzionali del sistema da sviluppare e studio delle tecnologie e linguaggi disponibili riguardanti la realtà virtuale;
	
	\item \textbf{Settimana 2:} in base ai risultati ottenuti nella prima settimana, viene richiesta una scelta dell'hardware da utilizzare e un \textit{framework}\ped{\hyperlink{fw}{G}} di sviluppo, testandoli con un primo prototipo di scena 3D;
	
	\item \textbf{Settimana 3:} previste attività di raffinamento della scena 3D, progettazione e sviluppo degli oggetti e loro comportamento nello spazio 3D. Viene creato così un primo prototipo di \textit{user interaction};
	
	\item \textbf{Settimana 4:} previste attività di progettazione e sviluppo integrazione tra sistema \textit{VR}\ped{\hyperlink{vr}{G}} e \textit{e-commerce}. Progettazione di \textit{user interaction} per la fruizione dei contenuti provenienti dall'\textit{e-commerce};
	
	\item \textbf{Settimana 5:} settimana dedicata all'approfondimento di \textit{user interaction} e del comportamento degli oggetti nell'ambiente virtuale;
	
	\item \textbf{Settimana 6:} previste attività di studio e prototipazione del possibile processo d'acquisto all'interno dell'ambiente virtuale;
	
	\item \textbf{Settimana 7:} la settima settimana rappresenta una \textit{milestone} importate per il progetto: conclusione del prototipo e relativa documentazione, raggiungendo così gli obbiettivi minimi;
	
	\item \textbf{Settimana 8:} l'ultima settimana viene dedicata completamente allo studio del modello emergente \textit{omni-channel}\ped{\hyperlink{oc}{G}} e come la realtà virtuale possa estendere questo modello. Vengono così raggiunti gli obbiettivi massimi.
\end{itemize}

\subsubsection{Piano metodologico}
	
Assieme al tutor aziendale, abbiamo fin da subito concordato la mia presenza durante l'orario d'ufficio, permettendo così un interazione intensa e costante. \\
Il lavoro di ricerca e sviluppo che ho effettuato è stato totalmente autonomo, con giornaliere interazioni con il personale solo per raccogliere e analizzare la documentazione, requisiti e \textit{feedback} sull'andamento del progetto. \\
Le revisioni di progetto sono avvenute secondo la seguente metodologia:

\begin{itemize}
	\item Riunione breve di 15 minuti ogni mattina;
	\item Riunione di 1 ora alla fine di ogni settimana come analisi retrospettiva.
\end{itemize}

Alle revisioni, oltre a me, hanno partecipato:

\begin{itemize}
	\item Valentino Baraldo, \textit{cloud engineer} e tutor aziendale. Oltre a svolgere il compito di tutor aziendale, mi ha supportato sulla progettazione architetturale del progetto e sull'utilizzo del servizio \textit{API Gateway} di \textit{Amazon Web Services}\footnote[1]{\url{https://aws.amazon.com/it/}};
	\item Francesco Paggin, \textit{front-end developer}. Ha supervisionato il mio lavoro grafico nell'ambiente di sviluppo \textit{Unity}\footnote[2]{\url{https://unity3d.com/}}.
\end{itemize}  

\subsubsection{Piano tecnologico}

Inizialmente lo \textit{stack} tecnologico propostomi riguardava solamente l'hardware che l'azienda aveva acquistato per questo progetto, senza alcun vincolo software. I dispositivi che permettevano la sperimentazione \textit{VR}\ped{\hyperlink{vr}{G}} erano:

\begin{itemize}
	\item \textbf{Oculus Rift Development kit 2:} visore per la realtà virtuale per uso desktop. Possiede uno schermo Samsung OLED 2160x1200 pixel (1080x1200 per occhio), con un \textir{refresh rate} a 90 Hz e un ampio angolo di visione a 110 gradi. Dotato di accelerometro, giroscopio, magnetometro e \textit{tracking} posizionale a 360 gradi. Viene accoppiato ad una telecamera infrarossi per il rilevamento di profondità, assieme a 40 emettitori infrarossi all'interno dell'\textit{headset}. Monta due lenti in alta definizione possedendo 6 gradi di libertà di rotazione;
	
	\item \textbf{Samsung Gear VR:} visore per la realtà virtuale per \textit{mobile}. Possiede: accelerometro, giroscopio e sensore di prossimità, permettendo un campo visivo di 96 gradi. Il visore incorpora inoltre un'interfaccia utente fisica: \textit{touch pad}, tasto indietro e tasto per il volume. Necessita l'inserimento di uno \textit{smartphone} Samsung a partire dalla versione \textit{Galaxy S6};
	
	\item \textbf{Google Cardboard:} con il termine \textit{Google Cardboard} non si intende specificare un particolare visore per la realtà virtuale prodotto fisicamente da \textit{Google}, ma un insieme di linee guida suggerite da questa per costruire un dispositivo a basso costo per l'uso mobile. In azienda erano presenti due visori che implementavano tali linee guida: \textit{Unofficial Cardboard} e \textit{Tera VR Box};
	
	\item \textbf{Leap Motion:} piccola periferica USB  progettata per essere posta su una scrivania reale rivolta verso l'alto. Usando 2 telecamere e 3 LED infrarossi essa osserva un'area approssimativamente a forma di semisfera di circa un metro. E' progettata per identificare dita (o oggetti simili come una penna) con una precisione di 0,01 mm. 
\end{itemize}

\label{Gear VR}
\begin{figure}[ht]
	\begin{center}
		\includegraphics[scale=0.15]{gearvr}
		\caption{Samsung Gear VR}
	\end{center}
\end{figure}
\FloatBarrier

Dopo un periodo di ricerca e \textit{testing} su queste tecnologie, abbiamo deciso di intraprendere la strada \textit{mobile}, a discapito di quella desktop. Questa decisione è stata dettata principalmente da due fattori:

\begin{itemize}
	\item \textbf{Requisiti hardware elevati}: per poter offrire un esperienza fluida e piacevole, \textit{Oculus Rift Development kit 2} abbisogna di un PC dall'hardware elevato, non accessibile all'utenza media:
	\begin{itemize}
		\item \textbf{GPU:} NVIDIA GTX 970 / AMD R9 290;
		\item \textbf{CPU:} Intel i5-4590;
		\item \textbf{RAM:} 8GB;
		\item \textbf{Video output:} HDMI 1.3;
		\item \textbf{USB Ports:} 3 porte 3.0 più una porta 2.0;
		\item \textbf{OS:} Windows 7 SP1 64 bit o superiore.
	\end{itemize}
	
	\item \textbf{Obiettivi aziendali:} anche se fin da subito mi era stato chiarito che non veniva preteso alcun prodotto finale utilizzabile, l'azienda sperava però di riuscire con questo stage di avere un primo prototipo di \textit{virtual showroom} da poter mostrare alle fiere tecnologiche alle quali partecipa. In quest'ottica, l'utilizzo di \textit{Oculus Rift Development kit 2} sarebbe risultato troppo scomodo sia per il trasporto e l'installazione, che per l'utilizzatore finale.
\end{itemize}

Si è deciso dunque di sviluppare sia per \textit{Samsung Gear VR} che per \textit{Google Cardboard}, entrambi dispositivi \textit{mobile} a costo contenuto. \\
Lo stack software che si è andato a formare poi, presa questa decisione, ha avuto un tempo decisionale assai breve. Dopo alcune ricerca da me effettuate, ho potuto riscontrare come \textit{Unity} fosse l'unico \textit{framework}\ped{\hyperlink{fw}{G}} per lo sviluppo \textit{VR}\ped{\hyperlink{vr}{G}} gratuito e ricco di documentazione, cosa estremamente fondamentale data la mia iniziale totale ignoranza sull'argomento.

\label{Unity}
\begin{figure}[ht]
	\begin{center}
		\includegraphics[scale=0.3]{unity}
		\caption{Logo Unity}
	\end{center}
\end{figure}
\FloatBarrier

\textit{Unity} è uno strumento di \textit{authoring}\ped{\hyperlink{auth}{G}} integrato e multipiattaforma per la creazione di videogiochi 3D o altri contenuti interattivi, quali visualizzazioni architettoniche o animazioni 3D in tempo reale. Esso, tramite gli \textit{SDK}\ped{\hyperlink{sdk}{G}} forniti da Samsung, per il dispositivo Gear VR, da Google, per il dispositivo Cardboard, e da Andriod, per l'interfacciamento con lo smartphone, permette lo sviluppo di applicazioni \textit{VR}\ped{\hyperlink{vr}{G}} fornendo un interfaccia per modellare "a mano" gli oggetti e i linguaggi JavaScript e C\# per sviluppare il loro comportamento. \\
Il linguaggio di \textit{scripting} da me scelto per modellare il comportamento degli oggetti nello spazio tridimensionale è C\# poiché è risultato essere il linguaggio più utilizzato nella \textit{community} \textit{Unity}.

\label{Stack Gear VR}
\begin{figure}[ht]
	\begin{center}
		\includegraphics[scale=0.28]{stack-gearvr}
		\caption{Stack tecnologico di sviluppo per Samsung Gear VR}
	\end{center}
\end{figure}
\FloatBarrier

\subsection{Obiettivi aziendali}

Nel \textit{Piano di Lavoro} presentatomi, l'azienda espone gli obbiettivi minimi e massimi che si aspetta di veder raggiunti alla fine delle 320 ore si stage:

\begin{itemize}
	\item \textbf{Obbiettivi minimi:}
	\begin{enumerate}
		\item Studio delle tecnologie disponibili in ambito \textit{VR}\ped{\hyperlink{vr}{G}} e stesura di un documento riassuntivo che offra un \textit{overview} dello stato attuale della realtà aumentata;
		\item Progettazione e sviluppo di un ambiente virtuale con: una scena e oggetti definiti, un comportamento associato agli oggetti, un prototipo di \textit{user interaction} e scambio di informazioni di base con un sistema di \textit{e-commerce}.
	\end{enumerate}
	\item \textbf{Obbiettivi massimi:}
	\begin{enumerate}
		\item Studio e prototipazione di diversi modelli di \textit{user interaction} con l'ambiente e con gli oggetti finalizzati alla presentazione di un bene vendibile;
		\item Studio e implementazione di possibili nuovi processi di acquisto in ambito \textit{VR}\ped{\hyperlink{vr}{G}}. 
	\end{enumerate}
\end{itemize}

\subsection{Obiettivi personali}

Sono venuto a conoscenza di questo progetto durante l'evento di Stage IT 2016, organizzato da Confindustria Padova in collaborazione con l'Università di Padova e Venezia. Mi ha fin da subito colpito e appassionato per le tecnologie che mi avrebbe permesso di studiare, come ad esempio \textit{Unity}. La computer grafica è da sempre un mio personale interesse e la realtà aumentata è un ambito per me molto affascinante e ricco di opportunità.

\label{Stage IT 2016}
\begin{figure}[ht]
	\begin{center}
		\includegraphics[scale=0.45]{stageit}
		\caption{Logo dell'evento Stage IT 2016}
	\end{center}
\end{figure}
\FloatBarrier

Le tecnologie proposte dall'azienda purtroppo non rientrano nel percorso di studi, dunque gli obbiettivi formativi personali che mi sono posto riguardano lo studio e la sperimentazione delle tecnologie, senza pretendere di arrivare ad un risultato non prototipale:

\begin{itemize}
	\item \textbf{Obiettivi minimi:}
	\begin{enumerate}
		\item Conoscenza ad alto livello delle tecnologie (hardware e software) attualmente disponibili nel mercato atte a creare ambienti virtuali;
		\item Conoscenza ad alto livello dei concetti principali di \textit{e-commerce} e relative tecnologie di riferimento usate per la vendita online.
	\end{enumerate}
	\item \textbf{Obiettivi massimi:}
	\begin{enumerate}
		\item Capacità di identificare, progettare e sviluppare ambienti virtuali, selezionando le tecnologie attualmente disponibili più appropriate per il caso d'uso;
		\item Presa di coscienza dei concetti multi-channel e omni-channel e come le nuove modalità di vendita si integrino con questi modelli emergenti.
	\end{enumerate}
\end{itemize}