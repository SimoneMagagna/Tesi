\newpage
%**************************************************************
\chapter{The White Dog s.r.l.}
\label{cap:thewhitedog}

\section{Chi è The White Dog s.r.l.}

The White Dog s.r.l. è una realtà aziendale nata il ... con sede a Torreglia, in provincia di Padova. Essa è stata fondata dal signor Stefano Mocellini, fondatore e CEO di Diana Corp., con la volontà di formare un \textit{team} che si dedichi completamente alla ricerca e sviluppo per quest'ultima.

\section{Prodotti e servizi}

Il principale servizio che l'azienda offre a Diana Corp. è la ricerca e lo sviluppo di nuove tecnologie da applicare nell'ambito del fashion e-commerce. Essa svolge l'attività di \textit{testing} delle nuove tecnologie web disponibili, le valuta attentamente in termini di prestazioni e costi, per poi renderle disponibili all'azienda Diana Corp.. Ad essa oltretutto vengono commissionati progetti che Diana Corp., per competenze e tempistiche, non può portare a termine, come ad esempio applicazioni \textit{mobile} legate agli e-commerce prodotti. \\ \\
Il prodotto cardine però dell'azienda è sicuramente \textit{Live Story}. \textit{Live Story} è un'applicazione web che permette alle aziende di moda di ricercare nei \textit{social network} foto, marcate con un particolare \textit{hashtag}, di utenti che indossano loro capi di abbigliamento e di pubblicarle così nel proprio sito/e-commerce. Questa applicazione ha trovato sin da subito largo interesse e consenso tra i \textit{brand} per i quali Diana Corp. già offriva servizi e-commerce, poiché permette di ottenere foto pubblicitarie pubbliche da utilizzare immediatamente, dopo ovviamente una fase di selezione accurata da parte di un operatore.

\section{Processi interni}

Lo sviluppo del software a The White Dog s.r.l. segue una metodologia tipicamente Agile. Questa metodologia permette all'azienda di rispondere in tempi brevi ai continui nuovi bisogni di Diana Corp., anche lei fortemente legata a questo metodo di lavoro. Essendo The White Dog s.r.l. formata da un \textit{team} composto da poche persone, tale metodo di lavoro risulta essere molto efficiente. \\
Le procedure, gli strumenti e le metriche adottate in The White Dog s.r.l. derivano da tre principali concetti di sviluppo Agile:

\subsubsection{DevOps}

Metodologia di sviluppo software che punta alla comunicazione, collaborazione e integrazione tra gli sviluppatori e addetti alle \textit{operations} dell'\textit{information technology}. DevOps vuole rispondere all'interdipendenza tra sviluppo software e IT \textit{operations}, puntando ad aiutare un'organizzazione a sviluppare in modo più rapido ed efficiente prodotti e servizi. \\ \\
In The White Dog s.r.l. questo principio è concretizzato dal fatto che ogni membro possiede sia le competenze di sviluppo, sia amministrative che di controllo della qualità, migliorando così di molto l'efficienza e l'agilità nello sviluppo del software e nel suo rilascio.

\subsubsection{Extreme Programming}

Metodologia di sviluppo software che enfatizza la scrittura di codice di qualità e la rapidità di risposta ai cambiamenti di requisiti. Prescrive lo sviluppo iterativo e incrementale soprattutto in brevi cicli di sviluppo. Suggerisce inoltre l'uso sistematico di \textit{unit testing} e \textit{refactoring}, vietando ai programmatori di sviluppare codice non strettamente necessario. Sostiene la chiarezza e la semplicità del codice, preferisce strutture gestionali non gerarchiche e dà molta importanza  alla comunicazione diretta e frequente fra sviluppatori e cliente e fra gli sviluppatori stessi. \\ \\
Il \textit{team} di sviluppo di The White Dog s.r.l. fa ampio utilizzo di questa metodologia, spingendo molto sulla semplicità del codice prodotto, che dovrà poi essere utilizzato dagli sviluppatori Diana Corp., e sulla giornaliera comunicazione diretta tra gli sviluppatori e con il loro principale cliente, ovvero Diana Corp.. Questa comunicazione è facilitata dal fatto che The White Dog s.r.l. ha sede nello stesso stabilimento di Diana Corp..

\subsubsection{Kanban}

Metodologia atta al miglioramento dei processi per garantire una produzione \textit{Just in Time}. Essa prevede la presa di coscienza del proprio flusso di lavoro, visualizzandolo all'interno di una lavagna fisica, formata da tante colonne quante sono le fasi del processo produttivo. \\ \\
All'interno dello studio di The White Dog s.r.l. un interno muro bianco è dedicato alla \textit{kanban board}, dove il team prima di ogni sviluppo crea un nuovo \textit{workflow} per visualizzare le attività da fare e assegnarle agli sviluppatori.

\section{Strumenti e tecnologie}

All'interno di questa sezione parlerò degli strumenti e delle tecnologie adottate in azienda per lo sviluppo software.

\section{Ricerca e innovazione}

The White Dog s.r.l. nasce come reparto di ricerca e sviluppo di Diana Corp.. Essa dunque sperimenta e studia ogni giorno nuove tecnologie applicabili nel mondo del fashion e-commerce. \\
Ha a disposizione diversi dispositivi per la ricerca come \textit{smartphone} di ultima generazione, \textit{Smart TV}, \textit{smartwatch} e numerosi dispositivi per lo sviluppo AR e VR come \textit{Google Glass}, \textit{Oculus Rift Development Kit 2}, Google Cardboard e Leap Motion. Attraverso questi dispositivi l'azienda studia e sviluppa nuove modalità di interazione che l'utente finale può utilizzare nell'acquisto nei propri \textit{store} digitali.